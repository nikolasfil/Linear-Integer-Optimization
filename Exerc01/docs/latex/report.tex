\documentclass[12pt]{report}
\usepackage[utf8]{inputenc}
\usepackage[T1 , LGR]{fontenc}
\usepackage[english , greek]{babel}
\usepackage[a4paper , total={7in,9in}]{geometry}
\usepackage{graphicx}
\usepackage{amsmath}
\usepackage{listings}
\usepackage{color}

\graphicspath{{./img/}}
\definecolor{dkgreen}{rgb}{0,0.6,0}
\definecolor{gray}{rgb}{0.5,0.5,0.5}
\definecolor{mauve}{rgb}{0.58,0,0.82}
\lstset{frame=tb,
    language=Python,
    aboveskip=3mm,
    belowskip=3mm,
    showstringspaces=false,
    columns=flexible,
    basicstyle={\small\ttfamily},
    numbers=none,
    numberstyle=\tiny\color{gray},
    keywordstyle=\color{blue},
    commentstyle=\color{dkgreen},
    stringstyle=\color{mauve},
    breaklines=true,
    breakatwhitespace=true,
    tabsize=4
}
\usepackage[figurename=Σχήμα]{caption}
\begin{document}
\selectlanguage{greek}
\author{Νικόλας Φιλιππάτος \\ ΑΜ:1072754}
\title{\textlatin{HW1 Linear Optimization 2022-23}}
\date{}
\maketitle
\section*{Άσκηση 1}
Έστω το πρόβλημα γραμμικού προγραμματισμού \\
\[
    \begin{array}{r@{}r@{}r@{}l}    \\
        \text{(Π1)} & {} 2x_1 + x_2 \geq 4               \\
        \text{(Π2)} & {} x_1 + 2x_2 \geq 5               \\
        \text{(Π3)} & {} x_1 - 2x_2 \leq 1               \\
                    & {} {x_1 , x_2 \geq 0}
    \end{array}
\] \\
(α) Παραστήστε γραφικά την εφικτή περιοχή του προβλήματος καθώς και όλες τις
κορυφές της. Περιγράψτε τη μορφή της εφικτής περιοχής. \\ \newline
(β) Λύστε το παραπάνω πρόβλημα γραφικά με κάθε μία από τις παρακάτω αντικειμενι-
κές συναρτήσεις: \\ \newline
(i) $ \text{\textlatin{max Z}}  = 2x_1 -  5x_2$ (ii) $ \text{\textlatin{max Z}}  = 2x_1 -  4x_2$ 
(iii) $ \text{\textlatin{max Z}}  = 2x_1 -  3x_2$ \\ \newline
Σε κάθε περίπτωση περιγράψτε αναλυτικά τη μορφή της λύσης, εφ όσον υπάρχει.

\subsection*{(a)}

\begin{figure}
    \centering
    \includegraphics[scale = 0.4]{exerc01-feasible_region-small.png}
    \caption{Γραφική απεικόνιση του συστήματος}
    \label{fig:1_a}    
\end{figure}

Αρχικά θα σχεδιάσουμε τις ευθείες των περιορισμών (Σχήμα
\ref{fig:1_a}). 
Η περιοχή εφικτών λύσεων του προβλήματος
βρίσκεται πάνω από τις ευθείες και είναι η γραμμοσκιασμένη περιοχή.




\section*{Άσκηση 2}
Έστω το πρόβλημα γραμμικού προγραμματισμού \\
% \[
%     \begin{array}{r@{}r@{}r@{}l}    \\
%         \text{(Π1)} & {} 2x_1 + x_2 \geq 4               \\
%         \text{(Π2)} & {} x_1 + 2x_2 \geq 5               \\
%         \text{(Π3)} & {} x_1 - 2x_2 \leq 1               \\
%                     & {} {x_1 , x_2 \geq 0}
%     \end{array}
% \] \\
% (α) Παραστήστε γραφικά την εφικτή περιοχή του προβλήματος καθώς και όλες τις
% κορυφές της. Περιγράψτε τη μορφή της εφικτής περιοχής. \\ \newline
% (β) Λύστε το παραπάνω πρόβλημα γραφικά με κάθε μία από τις παρακάτω αντικειμενι-
% κές συναρτήσεις: \\ \newline
% (i) $ \text{\textlatin{max Z}}  = 2x_1 -  5x_2$ (ii) $ \text{\textlatin{max Z}}  = 2x_1 -  4x_2$ 
% (iii) $ \text{\textlatin{max Z}}  = 2x_1 -  3x_2$ \\ \newline
% Σε κάθε περίπτωση περιγράψτε αναλυτικά τη μορφή της λύσης, εφ  όσον υπάρχει.

\subsection*{(a)}




% \subsection*{(β)}

% \begin{figure}[h]
%     % \centering
%     \includegraphics[width=\textwidth]{q1_b.png}
%     \caption{Η αντικειμενική συνάρτηση για $c_2 = 0.5$ και $ c_2 = 2$.}
%     \label{fig:1_b}
% \end{figure}
% Για το ερώτημα β, η αντικειμενική συνάρτηση γίνεται $ \text{\textlatin{min}} Z
%     = x_1 + c_2x_2 $ και αν λύσουμε
% ως προς $x_2$ έχουμε: $$ x_2 = \frac{Z}{c_2} - \frac{1}{c_2} x_1 $$
% Πρέπει να βρούμε το εύρος τιμών του $c_2$ έτσι ώστε η βέλτιστη λύση να είναι η
% τομή των ευθειών που ορίζουν οι
% περιορισμοί (Π1) και (Π2). Το σημείο τομής των ευθειών αυτών είναι το
% $(\frac{4}{3}, \frac{4}{3})$. Όπως
% βλέπουμε και στο Σχήμα \ref{fig:1_b}, για τιμές του $c_2 = 0.5 , c_2 = 2$ η
% αντικειμενική συνάρτηση ταυτίζεται
% με τις ευθείες των περιορισμών.

% Στην περίπτωση που μειώσουμε το $c_2$ από $0.5$ σε $0.43$ ή αυξήσουμε από $2$
% σε $2.15$ (Σχήμα \ref{fig:1_b2})
% παρατηρούμε πως οι ευθείες της αντικειμενικής συνάρτησης τέμνουν την περιοχή
% εφικτών λύσεων σε περισσότερα
% σημεία επομένως δεν έχουμε μοναδική βέλτιστη λύση. Όμως αν αυξήσουμε το $c_2$
% από $0.5$ σε $0.575$ ή μειώσουμε
% από $2$ σε $1.85$ (Σχήμα \ref{fig:1_b2}), βλέπουμε πως οι ευθείες της
% αντικειμενικής συνάρτησης τέμνουν την περιοχή
% εφικτών λύσεων μόνο στο σημείο $(\frac{4}{3}, \frac{4}{3})$.
% Επομένως το εύρος τιμων που μπορεί να πάρει το $c_2$ είναι μεταξύ του 0.5 και
% 2. Παρατηρούμε πως:
% $$ 0.5 < c_2 < 2 \Leftrightarrow \frac{1}{2} < c_2 < \frac{2}{1}
%     \Leftrightarrow
%     \frac{2}{1} > \frac{1}{c_2} > \frac{1}{2} \Leftrightarrow -2 <
%     -\frac{1}{c_2} <
%     -\frac{1}{2}$$
% όπου -2 και $-\frac{1}{2}$ είναι οι κλίσεις των (Π1) και (Π2) αντίστοιχα. Άρα
% πρέπει η κλίση της αντικειμενικής
% συνάρτησης να βρίσκεται ανάμεσα στις κλίσεις των περιορισμών (Π1) και (Π2)
% \begin{figure}[h]
%     % \centering
%     \includegraphics[width=\textwidth]{q1_b1.png}
%     \caption{Η αντικειμενική συνάρτηση για $c_2 = 0.43$ και $c_2 = 2.15$.}
%     \label{fig:1_b1}
% \end{figure}

% Δηλαδή $0.5 < c_2 < 2$, δεν συμπεριλαμβάνουμε το 0.5 και το 2, διότι τότε δεν
% θα είχαμε μόνο το
% σημείο $(\frac{4}{3}, \frac{4}{3})$ ως βέλτιστη λύση.

% \begin{figure}[!t]
%     % \centering
%     \includegraphics[width=\textwidth]{q1_b2.png}
%     \caption{Η αντικειμενική συνάρτηση για $c_2 = 0.575$ και $c_2 = 1.85$.}
%     \label{fig:1_b2}
% \end{figure}

% \subsection*{(γ)}
% Η αντικειμενική συνάρτηση του ερωτήματος γ είναι $\text{\textlatin{max}} Z =
%     c_1 x_1 + c_2 x_2$. Αν λύσουμε ως προς $x_2$:
% $$ x_2 = \frac{Z}{c_2} - \frac{c_1}{c_2}x_1 $$
% Αν εργαστούμε παρόμοια με το ερώτημα β θα πρέπει το $-\frac{c_1}{c_2}$ να
% βρίσκεται ανάμεσα στις κλίσεις των περιορισμών
% (Π3) και (Π4). Δηλαδή θα έχουμε:
% $$ -\frac{6}{5} < -\frac{c_1}{c_2} <-\frac{6}{7} \Leftrightarrow \frac{6}{7} >
%     \frac{c_1}{c_2} > \frac{6}{5}$$
% Παρακάτω μπορούμε να δούμε την νέα αντικειμενική συνάρτηση για διάφορες κλίσεις
% (Σχήμα \ref{fig:1_c} και \ref{fig:1_c1})

% \begin{figure}[h]
%     \includegraphics[width=\textwidth]{q1_c.png}
%     \caption{Η αντικειμενική συνάρτηση για κλίση ίδια με τους περιορισμούς (Π3)
%         και (Π4)}
%     \label{fig:1_c}
% \end{figure}
% \begin{figure}[h]
%     \includegraphics[width=\textwidth]{q1_c1.png}
%     \caption{Η αντικειμενική συνάρτηση για κλίση ίση με 1.1 και 0.9}
%     \label{fig:1_c1}
% \end{figure}
% Ο κώδικας που χρησιμοποιήθηκε για την άσκηση 1 βρίσκεται στο αρχείο
% \textlatin{question1.py}
% \clearpage
% Κώδικας για την άσκηση 1.
% \selectlanguage{english}
% \lstinputlisting[language=Python]{src/question1.py}
% \selectlanguage{greek}
% \clearpage
% \section*{Άσκηση 2}
% Θεωρήστε το παρακάτω τοπικό δίκτυο, το οποίο πρόκειται να χρησιμοποι-
% ήσει κάποιος για να μεταφέρει τη μουσική συλλογή που διατηρεί σε έναν παλιό
% υπολογιστή
% (κόμβος $s$) στον καινούργιο υπολογιστή (κόμβος $t$).\\
% \begin{figure}[h]
%     \includegraphics[width=\textwidth]{q2_a.png}
% \end{figure}
% \\
% Οι αριθμοί δίπλα σε κάθε σύνδεσμο μεταξύ κόμβων δηλώνουν την μέγιστη δυνατή ροή
% δεδομένων πάνω στον σύνδεσμο. Η ροή σε κάθε σύνδεσμο μπορεί να γίνει είτε προς
% τη μία
% είτε προς την άλλη κατεύθυνση, αλλά ποτέ και στις δύο ταυτόχρονα. Έτσι για
% παράδειγμα,
% διά μέσου του συνδέσμου (a, b) θα μπορούσαμε να στείλουμε δεδομένα μέχρι 1M
% bit/s είτε
% από τον a στον b είτε από τον b στον a. Οι κόμβοι a, b, c, d, e δεν είναι
% σχεδιασμένοι για να
% αποθηκεύουν δεδομένα κι επομένως όλα τα δεδομένα που εισέρχονται σε αυτούς θα
% πρέπει
% να προωθηθούν αμέσως παρακάτω. Ο στόχος είναι να βρεθεί η βέλτιστη ροή
% δεδομένων
% που μπορούμε να έχουμε πάνω σε κάθε σύνδεσμο του δικτύου έτσι ώστε η συνολική
% ροή
% να είναι η μέγιστη. Μοντελοποιήστε το πρόβλημα που περιγράψαμε ώς πρόβλημα
% γραμμικού
% προγραμματισμού.\\
% {[Υπόδειξη. Για οικονομία στις μεταβλητές χρησιμοποιήστε μία μεταβλητή για κάθε
%         σύνδεσμο, η οποία μπορεί να πάρει θετικές ή αρνητικές τιμές. Έτσι στη
%         λύση μία
%         θετική
%         τιμή θα δηλώνει την μία κατεύθυνση ενώ μία αρνητική την αντίθετη της]}
% \subsection*{(α)}
% Αρχίκα ορίζουμε $x_{i,j}$ την ροή από το σημείο $i$ στο σημείο $j$ σε
% \textlatin{Mbit/s}. Ακόμα γνωρίζουμε ότι
% $x_{i,j} = x_{j,i}$ αλλά για να δηλώσουμε και την κατεύθυνση θα είναι $x_{i,j}
%     = -x_{j,i}$\\\\
% Από το σχήμα έχουμε τους παρακάτω περιορισμούς:
% $$
%     \begin{array}{rrrrr}
%         |x_{e,t} |\leq 1, & |x_{a,d} |\leq 1, & |x_{a,b} |\leq 1, & |x_{d,t}
%         |\leq 4,
%                           & |x_{s,c} |\leq 1,
%         \\
%         |x_{c,e} |\leq 4, & |x_{c,d} |\leq 4, & |x_{s,b} |\leq 1, & |x_{b,e}
%         |\leq 3,
%                           & |x_{s,a} |\leq 3,
%     \end{array}
% $$
% Έστω ότι, το $s$ στέλνει δεδομένα στο $a$, τότε το $a$ με την σειρά του θα
% μπορέι να στείλει δεδομένα στο $d$ και $b$.
% Τότε θα έχουμε ότι η ροή του $s$ στο $a$ θα είναι ίση με την ταχύτητα που
% μπορεί να στείλει ο $a$ στον $d$ και $b$. \\\\
% Επομένως:
% $$ x_{s,a} = x_{a,b} + x_{a,d}$$
% Με την ίδια λογική θα καταλήξουμε στο:
% $$
%     \begin{array}{r}
%         x_{s,a} = x_{a,b} + x_{a,d} \\
%         x_{s,b} = x_{b,a} + x_{b,e} \\
%         x_{s,c} = x_{c,d} + x_{c,e} \\
%         x_{e,t} = x_{b,e} + x_{c,e} \\
%         x_{d,t} = x_{c,d} + x_{a,d}
%     \end{array}
% $$
% Τελικά η αντικειμενική συνάρτηση θα είναι η:
% $$ \text{\textlatin{max}} Z = x_{s,a} + x_{s,b} +x_{s,c} +x_{e,t} + x_{d,t} $$
% \clearpage
% \section*{Άσκηση 3}
% Επιχείρηση παραγωγής ενός εποχιακού και ευαίσθητου προϊόντος (π.χ. παγωτού ή
% αναψυκτικών) επιθυμεί να συντάξει ένα πρόγραμμα παραγωγής για τον επόμενο
% χρόνο. Το τμήμα πωλήσεων μετά από ενδελεχή έρευνα και λεπτομερή ιστορικά
% δεδομένα
% που έχει συλλέξει από προηγούμενα χρόνια προβλέπει ότι για τον επόμενο χρόνο η
% ζήτηση
% για κάθε μήνα $i$ θα είναι $d_i, i = 1, 2, . . . , 12$. Θεωρήστε τις μεταβλητές
% $x_i$ να είναι οι ποσότητες (σε $kg$) που θα παραχθούν κατά τη διάρκεια του
% μήνα $i$ όπως επίσης και τις μεταβλητές
% $s_i$ να είναι οι ποσότητες (σε $kg$) του αδιάθετου προϊόντος στο τέλος του
% μήνα $i$. Για ευκολία
% θα υποθέσουμε ότι $s_0 = 0$ (δεν υπάρχει αδιάθετο προϊόν από την προηγούμενη
% χρονιά) και
% $s{12} = 0$ (δεν θα μείνει αδιάθετο προϊόν στο τέλος της χρονιάς). Προφανώς η
% παραγωγή του
% προϊόντος θα παρουσιάζει αυξομειώσεις από μήνα σε μήνα και η επιχείρηση
% υπολογίζει ότι
% οι διαφορές στο ύψος της παραγωγής κάθε μήνα σε σχέση με τον προηγούμενο
% επιφέρει ένα
% κόστος 50 ευρώ ανά $kg$, ενώ το κόστος για την αποθήκευση των αδιάθετων
% προϊόντων από
% τον προηγούμενο μήνα κοστίζει 20 ευρώ ανά $kg$.\\
% Μοντελοποιήστε τα παραπάνω ώς πρόβλημα γραμμικού προγραμματισμού με
% αντικειμενικό σκοπό την ελαχιστοποίηση του συνολικού (ετήσιου) κόστους της
% επιχείρησης για την
% εκτέλεση του προγράμματος παραγωγής.
% \subsection*{(α)}
% Αρχική περιορισμοί:
% $$
%     \begin{array}{rl}
%         d_i \geq 0 , & i = 1..12 \\
%         x_i \geq 0 , & i = 1..12 \\
%         s_i \geq 0 , & i = 0..12 \\
%     \end{array}
% $$
% Για τον πρώτο μήνα ${(i = 1)}$ θα πρέπει η παραγωγή να ισούται με την
% προβλεπόμενη ζήτηση
% επειδή $s_0 = 0$ δηλαδή δεν υπάρχει απόθεμα, τότε το $ x_1 = d_1 $.\\
% Για τον δεύτερο μήνα η παραγωγή θα πρέπει να ισούτα με την προβλεπόμενη ζήτηση
% μείον το απόθεμα απο τον
% προηγούμενο μήνα, δηλαδή $ x_2 = d_2 - s_1 $.\\\\
% Επομένως η παραγωγή κάθε μήνα θα είναι $ x_i = d_i - s_{i-1}$.\\

% Ακόμα εμείς θέλουμε να ελαχιστοποιήσουμε το κόστος. Δηλαδή:
% $$
%     \text{\textlatin{min}}{(50 \sum^{n = 12}_{i=1}{(|x_{i-1} + x_i|)})}
% $$
% $$
%     \text{\textlatin{min}}{(20 \sum^{n = 12}_{i=2}{(s_i)})}
% $$

% \clearpage
% \section*{Άσκηση 4}
% Ένα σύνολο $X \subset R^n $ είναι κυρτό αν και μόνο αν \\
% $$ \forall x,y \epsilon \text{ } X \text{ και } \forall \lambda \epsilon
%     {[0,1]} : \lambda x+(1,-\lambda)y \text{ }\epsilon \text{ }X $$
% \subsection*{4.1}
% Έστω τα σημεία $x_1 , x_2 \text{ }\epsilon \text{ } X$\\
% $$
%     \begin{array}{l}
%         x_1 = (5 , 10) , x_2 = (9 , -100), \lambda = 0.5 \\
%         \lambda x_1 + (1 - \lambda) x_2 = (7 , -40)
%     \end{array}
% $$
% Επομένως το σύνολο δεν είναι κυρτό.
% \subsection*{4.2}
% Έστω τα σημεία $x_1 , x_2 \text{ }\epsilon \text{ } X$\\
% $$
%     \begin{array}{l}
%         x_1 = (1 , 1) , x_2 = (-1 , -1), \lambda = 0.5 \\
%         \lambda x_1 + (1 - \lambda) x_2 = (0 , 0)
%     \end{array}
% $$
% Επομένως το σύνολο δεν είναι κυρτό.
% \subsection*{4.3}
% Έστω τα σημεία $x , y \text{ }\epsilon \text{ } X.$ $x = {(x_1 , x_2)} , y =
%         {(y_1 , y_2)}$ \\
% Ισχύει ότι $x_1 , x_2 \text{ }\epsilon \text{ } X$ και $ y_1 , y_2 \text{
%     }\epsilon \text{ } {[3,4]}$
% $$
%     \begin{array}{lcl}
%         3 \leq y_1 \leq 4                         & {} & 3 \leq y_2 \leq 4
%         \\
%         3 \lambda \leq \lambda y_1 \leq 4 \lambda & {} & 3(1 - \lambda) \leq
%         (1- \lambda)
%         y_2 \leq 4 (1- \lambda)
%     \end{array}
% $$
% Προσθέτοντας τις δύο ανισώσεις θα έχουμε:
% $$ 3 \leq \lambda y_1 + (1- \lambda) y_2 \leq 4 $$
% Παρόμοια για το $x$ έχουμε:
% $$
%     \begin{array}{lcl}
%         -\infty \leq x_1 \leq + \infty        & {} & -\infty \leq x_2 \leq
%         +\infty                                                                \\
%         -\infty \leq \lambda x_1 \leq +\infty & {} & -\infty \leq (1- \lambda)
%         x_2 \leq
%         +\infty
%     \end{array}
% $$
% Προσθέτοντας:
% $$ -\infty \leq \lambda x_1 + (1- \lambda) x_2 \leq +\infty $$
% Από τις σχέσεις καταλήγουμε ότι:
% $$ \lambda x + (1 - \lambda) y \text{ }\epsilon \text{ } X$$
% Επομένως το σύνολο είναι κυρτό.
% \clearpage
% \section*{Άσκηση 5}
% Θεωρούμε το πρόβλημα γραμμικού προγραμματισμού:
% \[
%     \begin{array}{r@{}r@{}r@{}r@{}l}
%          & {}\text{\textlatin{max}}Z=4x_1 + 2x_2 + x_3          \\[\jot]
%         \text{όταν}                                             \\
%          & {} x_1  \leq                                & {} 5   \\
%          & {} 4x_1 + x_2 \leq                          & {} 25  \\
%          & {} 8x_1 + 4x_2 + x_3 \leq                   & {} 125 \\
%          & {} {x_1 , x_2 , x_3} \geq                   & {}  0
%     \end{array}
% \] \\
% (α) Προσθέστε μεταβλητές χαλάρωσης στο σύστημα ανισώσεων και βρείτε όλες τις
% βασικές (εφικτές και μη-εφικτές) λύσεις για το μη ομογενές σύστημα εξισώσεων
% που
% δημιουργείται. Εντοπίστε (αν υπάρχουν) τις εκφυλισμένες βασικές λύσεις. \\ \\
% (β) Βρείτε όλες τις κορυφές που δημιουργούνται από τις τομές των υπερεπιπέδων
% που
% αντιστοιχούν στις ανισώσεις/εξισώσεις του προβλήματος γραμμικού προγραμματισμού
% και ξεχωρίστε ποιες από αυτές είναι κορυφές του πολύτοπου των εφικτών λύσεων
% του.
% Εντοπίστε, αν υπάρχουν, τις εκφυλισμένες κορυφές.\\\\
% (γ) Βρείτε την αντιστοίχιση μεταξύ βασικών λύσεων και κορυφών και τέλος
% υποδείξτε
% τη βέλτιστη λύση και βέλτιστη κορυφή του προβλήματος. \\ \\
% \subsection*{(a)}
% Προσθέτοντας μεταβλητές χαλάρωσης θα έχουμε:
% \[
%     \begin{array}{r@{}r@{}r@{}r@{}l}
%          & {}\text{\textlatin{max}}Z=4x_1 + 2x_2 + x_3          \\[\jot]
%         \text{όταν}                                             \\
%          & {} x_1 + x_4 =                              & {} 5   \\
%          & {} 4x_1 + x_2 + x_5=                        & {} 25  \\
%          & {} 8x_1 + 4x_2 + x_3 + x_6=                 & {} 125 \\
%          & {} {x_1 , x_2 , x_3,x_4 ,x_5,x_6} \geq      & {}	0
%     \end{array}
% \] \\
% Άρα θα έχουμε τους πίνακες
% $$ B =
%     \begin{pmatrix}
%         1 & 0 & 0 & 1 & 0 & 0 \\
%         4 & 1 & 0 & 0 & 1 & 0 \\
%         8 & 4 & 1 & 0 & 0 & 1
%     \end{pmatrix}
% $$
% \begin{center}
%     και
% \end{center}
% $$ b =
%     \begin{pmatrix}
%         5 & 25 & 125
%     \end{pmatrix}
% $$

% Για να βρούμε όλες τις λύσεις του συστήματος θα πρέπει να πάρουμε όλους τους
% συνδυασμούς των
% μεταβλητών ανά τρεις. Οι τρεις μεταβλητές που θα διαλέγουμε κάθε φορά θα είναι
% οι βασικές μας μεταβλητές
% και οι υπόλοιπες θα είναι οι μη-βασικές τις οποίες θα θεωρούμε 0.
% Πιο συγκεκριμένα κάθε βασική λύση δίνεται από τον τύπο:
% $$ x_B = B^{-1} b \text{ και } x_N = 0 $$

% Με βάση το $x_B$ μπορούμε να διακρίνουμε κάποιες περιπτώσεις για την αξιολόγηση
% της λύσης.\\ \\
% 1) Αν $ x_B < 0 $, η λύση είναι βασική μη εφικτή. \\
% 2) Αν $ x_B \geq 0$ , τότε η λύση είναι βασίκη εφικτή λύση. \\
% 3) Αν $x_B > 0 $, τότε η βασική λύση είναι μη εκφυλισμένη, ενώ αν υπάρχει
% τουλάχιστον ένα μηδενικό
% στον πίνακα $ x_B $, ονομάζεται εκφυλισμένη.\\

% Βρέθηκαν συνολικά 14 λύσεις, εκ των οποίων 9 είναι εφικτές μη εκφυλισμένες και
% 5 μη εφικτές. Δεν βρέθηκε
% καμία εκφυλισμένη λύση.\\
% Λύσεις:
% $$\begin{array}{r@{}c@{}l@{}l@{}}
%         B_1 = \begin{Bmatrix}x_1 &x_2 & x_3 \end{Bmatrix} ,\text{ }    &
%         x_{B_1} =
%                                                       & \begin{pmatrix} 5.00 &
%                 5.00 & 65.00\end{pmatrix} & \text{Eφικτή μη
%             εκφυλισμένη}
%         \\
%         B_2 = \begin{Bmatrix}x_1 &x_2 & x_4 \end{Bmatrix} ,\text{ }    &
%         x_{B_2} =
%                                                       & \begin{pmatrix} -3.125
%              & 37.5 & 8.125\end{pmatrix} & \text{Eφικτή μη
%             εκφυλισμένη}
%         \\
%         B_3 = \begin{Bmatrix}x_1 &x_2 & x_5 \end{Bmatrix} ,\text{ }    &
%         x_{B_3} =
%                                                       & \begin{pmatrix} 5.00
%              & 21.25 & -16.25\end{pmatrix} & \text{Μη εφικτή}             \\
%         B_4 = \begin{Bmatrix}x_1 &x_2 & x_6 \end{Bmatrix} ,\text{ }    &
%         x_{B_4} =
%                                                       & \begin{pmatrix} 5.00 &
%                 5.00 & 65.\end{pmatrix} & \text{Eφικτή μη
%             εκφυλισμένη}
%         \\
%         B_5 = \begin{Bmatrix}x_1 &x_3 & x_4 \end{Bmatrix} ,\text{ }    &
%         x_{B_5} =
%                                                       & \begin{pmatrix} 6.25
%              & 75. & -1.25\end{pmatrix} & \text{Μη εφικτή}             \\
%         B_6 = \begin{Bmatrix}x_1 &x_3 & x_5 \end{Bmatrix} ,\text{ }    &
%         x_{B_6} =
%                                                       & \begin{pmatrix} 5. & 85.
%                    & 5.\end{pmatrix} & \text{Eφικτή μη εκφυλισμένη} \\
%         B_7 = \begin{Bmatrix}x_1 &x_4 & x_5 \end{Bmatrix} ,\text{ }    &
%         x_{B_7} =
%                                                       & \begin{pmatrix} 15.625
%              & -10.625 & -37.5\end{pmatrix} & \text{Μη εφικτή}             \\
%         B_8 = \begin{Bmatrix}x_1 &x_4 & x_6 \end{Bmatrix} ,\text{ }    &
%         x_{B_8} =
%                                                       & \begin{pmatrix} 6.25
%              & -1.25 & 75.\end{pmatrix} & \text{Μη εφικτή}             \\
%         B_9 = \begin{Bmatrix}x_1 &x_5 & x_6 \end{Bmatrix} ,\text{ }    &
%         x_{B_9} =
%                                                       & \begin{pmatrix} 5. &
%                 5. & 85.\end{pmatrix} & \text{Eφικτή μη εκφυλισμένη} \\
%         B_{10} = \begin{Bmatrix}x_2 &x_3 & x_4 \end{Bmatrix} ,\text{ } &
%         x_{B_{10}} =
%                                                       & \begin{pmatrix} 25.
%              & 25. & 5.\end{pmatrix} & \text{Eφικτή μη εκφυλισμένη} \\
%         B_{11} = \begin{Bmatrix}x_2 &x_4 & x_5 \end{Bmatrix} ,\text{ } &
%         x_{B_{11}} =
%                                                       & \begin{pmatrix} 31.25 &
%                 5.    & -6.25\end{pmatrix} & \text{Μη εφικτή}             \\
%         B_{12} = \begin{Bmatrix}x_2 &x_4 & x_6 \end{Bmatrix} ,\text{ } &
%         x_{B_{12}} =
%                                                       & \begin{pmatrix}25. &
%                5.  & 25.\end{pmatrix} & \text{Eφικτή μη εκφυλισμένη} \\
%         B_{13} = \begin{Bmatrix}x_3 &x_4 & x_5 \end{Bmatrix} ,\text{ } &
%         x_{B_{13}} =
%                                                       & \begin{pmatrix} 125. &
%                 5.   & 25.\end{pmatrix} & \text{Eφικτή μη
%             εκφυλισμένη}
%         \\
%         B_{14} = \begin{Bmatrix}x_4 &x_5 & x_6 \end{Bmatrix} ,\text{ } &
%         x_{B_{14}} =
%                                                       & \begin{pmatrix}  5.  &
%                  25. & 125.\end{pmatrix} & \text{Eφικτή μη εκφυλισμένη} \\
%     \end{array}$$
% \subsection*{(β)}
% Για να βρούμε τις κορυφές που δημιουργούνται από τις τομές των υπερεπιπέδων που
% αντιστοιχούν στις ανισώσεις/εξισώσεις του προβλήματος
% πρέπει να πάρουμε όλους τους συνδυασμούς των καμπύλων των περιορισμών ανά 3 και
% να βρούμε το σημείο τομής τους.\\
% Σημεία τομής:
% $$
%     \begin{array}{l@{}c@{}r@{}}
%         \text{Κορυφή } K_1 \text{ στο σημείο}    & \begin{pmatrix} 5.0 & 5.0 &
%                 65.0
%         \end{pmatrix} & \text{ είνα εφικτή.}    \\
%         \text{Κορυφή } K_2 \text{ στο σημείο}    & \begin{pmatrix} 5.0 & 5.0 &
%                 0.0
%         \end{pmatrix} & \text{ είνα εφικτή.}    \\
%         \text{Κορυφή } K_3 \text{ στο σημείο}    & \begin{pmatrix} 5.0 & 0.0 &
%                 85.0
%         \end{pmatrix} & \text{ είνα εφικτή.}    \\
%         \text{Κορυφή } K_4 \text{ στο σημείο}    & \begin{pmatrix} 5.0 & 21.5 &
%                 0.0
%         \end{pmatrix} & \text{ είνα μη εφικτή.} \\
%         \text{Κορυφή } K_5 \text{ στο σημείο}    & \begin{pmatrix} 5.0 & 0.0 &
%                 0.0
%         \end{pmatrix} & \text{ είνα εφικτή.}    \\
%         \text{Κορυφή } K_6 \text{ στο σημείο}    & \begin{pmatrix} 0.0 & 25.0 &
%                 25.0
%         \end{pmatrix} & \text{ είνα εφικτή.}    \\
%         \text{Κορυφή } K_7 \text{ στο σημείο}    & \begin{pmatrix} 6.25 & 0.0 &
%                 75.0
%         \end{pmatrix} & \text{ είνα μη εφικτή.} \\
%         \text{Κορυφή } K_8 \text{ στο σημείο}    & \begin{pmatrix} -3.125 &
%                 37.5   & 0.0
%         \end{pmatrix} & \text{ είνα μη εφικτή.} \\
%         \text{Κορυφή } K_9 \text{ στο σημείο}    & \begin{pmatrix} 0.0 & 25.0 &
%                 0.0
%         \end{pmatrix} & \text{ είνα εφικτή.}    \\
%         \text{Κορυφή } K_{10} \text{ στο σημείο} & \begin{pmatrix} 6.25 & 0.0 &
%                 0.0
%         \end{pmatrix} & \text{ είνα μη εφικτή.} \\
%         \text{Κορυφή } K_{11} \text{ στο σημείο} & \begin{pmatrix} 0.0 & 0.0 &
%                 125.0
%         \end{pmatrix} & \text{ είνα εφικτή.}    \\
%         \text{Κορυφή } K_{12} \text{ στο σημείο} & \begin{pmatrix} 0.0 & 31.25
%                     & 0.0
%         \end{pmatrix} & \text{ είνα μη εφικτή.} \\
%         \text{Κορυφή } K_{13} \text{ στο σημείο} & \begin{pmatrix} 15.625 & 0.0
%                        & 0.0
%         \end{pmatrix} & \text{ είνα μη εφικτή.} \\
%         \text{Κορυφή } K_{14} \text{ στο σημείο} & \begin{pmatrix} 0.0 & 0.0 &
%                 0.0
%         \end{pmatrix} & \text{ είνα εφικτή.}    \\
%     \end{array}
% $$
% \subsection*{(γ)}
% Επειδή δεν έχουμε εκφυλισμένες λύσεις, τότε η αντιστοίχιση των κορυφών με των
% λύσεων είναι μία προς μία. Για να κάνουμε
% την αντιστοίχιση χρησιμοποιούμε ως βάση τις τιμές των μεταβλητών $ x_1 , x_2 ,
%     x_3 $.
% $$
%     \begin{array}{l@{}c@{}l@{}}
%         K_1 \rightarrow B_1,       & \text{ εφικτή,}     & Z = 95   \\
%         K_2 \rightarrow B_4,       & \text{ εφικτή,}     & Z = 30   \\
%         K_3 \rightarrow B_6,       & \text{ εφικτή,}     & Z = 107  \\
%         K_4 \rightarrow B_3,       & \text{ μη εφικτή,}  & Z = 62.5 \\
%         K_5 \rightarrow B_9,       & \text{ εφικτή, }    & Z = 20   \\
%         K_6 \rightarrow B_{10},    & \text{ εφικτή,}     & Z = 95   \\
%         K_7 \rightarrow B_5,       & \text{ μη εφικτή,}  & Z = 100  \\
%         K_8 \rightarrow B_2,       & \text{ μη εφικτή,}  & Z = 62.5 \\
%         K_9 \rightarrow B_{12},    & \text{ εφικτή,}     & Z = 50   \\
%         K_{10} \rightarrow B_8,    & \text{ μη εφικτή,}  & Z = 25   \\
%         K_{11} \rightarrow B_{13}, & \text{ εφικτή, }    & Z = 125  \\
%         K_{12} \rightarrow B_{11}, & \text{ μη εφικτή, } & Z = 62.5 \\
%         K_{13} \rightarrow B_7,    & \text{ μη εφικτή,}  & Z = 62.5 \\
%         K_{14} \rightarrow B_{14}, & \text{ εφικτή,}     & Z = 0    \\
%     \end{array}
% $$
% Παίρνοντας υπόψην μόνο τις εφικτές λύσεις/κορυφές καταλήγουμε ότι η βέλτιση
% λύση/κορυφή είναι η
% $ (x_1 ,x_2 , x_3) = (0,0,125) $, και κορυφή η $K_11$. Η τιμή που παίρνει η
% αντικειμενική συνάρτηση
% είναι $ Z = 125 $\\
% \clearpage
% Κώδικας για την άσκηση 5.
% \selectlanguage{english}
% \lstinputlisting[language=Python]{src/question5.py}
% \selectlanguage{greek}
% \clearpage
% \section*{Άσκηση 6}
% Θεωρούμε το πρόβλημα γραμμικού προγραμματισμού:
% $$
%     \begin{array}{r@{}r@{}r@{}r@{}l}
%          & {}\text{\textlatin{max}}Z=4x_1 + 2x_2 + x_3          \\[\jot]
%         \text{όταν}                                             \\
%          & {} x_1  \leq                                & {} 5   \\
%          & {} 4x_1 + x_2 \leq                          & {} 25  \\
%          & {} 8x_1 + 4x_2 + x_3 \leq                   & {} 125 \\
%          & {} {x_1 , x_2 , x_3} \geq                   & {}  0
%     \end{array}
% $$
% α) Εφαρμόστε τον αλγόριθμο Simplex για να βρείτε τη βέλτιστη λύση του, αν
% υπάρχει.
% Σε κάθε επανάληψη του αλγορίθμου θα πρέπει να περιγράφετε συνοπτικά τα βήματα
% που ακολουθείτε και τις αποφάσεις που παίρνετε μέχρι το επόμενο βήμα.\\\\
% (β) Εφαρμόστε όλες τις εναλλακτικές επιλογές που μπορεί να έχετε σε κάθε βήμα
% επιλογής της εισερχόμενης ή εξερχόμενης μεταβλητής σε κάθε επανάληψη του αλγο-
% ρίθμου και δημιουργήστε έναν γράφο με τα βήματα (κορυφές) του αλγορίθμου μέχρι
% τη βέλτιστη λύση.\\\\

% \subsection*{(α)}
% $$
%     \begin{array}{r|rrrrrr|r}
%             & x_1 & x_2 & x_3 & x_4 & x_5 & x_6 &     \\ \hline
%         -Z  & 4   & 2   & 1   & 0   & 0   & 0   & 0   \\ \hline\hline
%         x_4 & 1   & 0   & 0   & 1   & 0   & 0   & 5   \\
%         x_5 & 4   & 1   & 0   & 0   & 1   & 0   & 25  \\
%         x_6 & 8   & 4   & 1   & 0   & 0   & 1   & 125 \\ \hline
%     \end{array}
% $$
% Για να επιλέξουμε ποια μεταβλητή θα αλλάξουμε κοιτάμε την γραμμή $-Z$ και
% βρίσκουμε τον μεγαλύτερο
% θετικό αριθμό, για το βήμα 1 η στήλη που διαλέγουμε είναι η $x_1$. Στην
% συνέχεια χρησιμοποιούμε το κριτήριο του
% λόγου για να βρόυμε ποια μεταβλητή θα αλλάξουμε, για το βήμα 1 είναι:
% $\text{\textlatin{min}}({\frac{5}{1} ,\frac{25}{4}, \frac{125}{8}}) =
%     \frac{5}{1}$ άρα διαλέγουμε την
% γραμή $x_4$\\\\
% Αλλάζουμε την μεταβλητή $x_4$ με την $x_1$.
% $$
%     \begin{array}{r|rrrrrr|r}
%             & x_1 & x_2 & x_3 & x_4 & x_5 & x_6 &     \\ \hline
%         -Z  & 0   & 2   & 1   & -4  & 0   & 0   & -20 \\ \hline\hline
%         x_1 & 1   & 0   & 0   & 1   & 0   & 0   & 5   \\
%         x_5 & 0   & 1   & 0   & -4  & 1   & 0   & 5   \\
%         x_6 & 0   & 4   & 1   & -8  & 0   & 1   & 85  \\ \hline
%     \end{array}
% $$
% Αλλάζουμε την μεταβλητή $x_5$ με την $x_2$.
% $$
%     \begin{array}{r|rrrrrr|r}
%             & x_1 & x_2 & x_3 & x_4 & x_5 & x_6 &     \\ \hline
%         -Z  & 0   & 0   & 1   & 4   & -2  & 0   & -30 \\ \hline\hline
%         x_1 & 1   & 0   & 0   & 1   & 0   & 0   & 5   \\
%         x_2 & 0   & 1   & 0   & -4  & 1   & 0   & 5   \\
%         x_6 & 0   & 0   & 1   & 8   & -4  & 1   & 65  \\ \hline
%     \end{array}
% $$
% Αλλάζουμε την μεταβλητή $x_1$ με την $x_4$.
% $$
%     \begin{array}{r|rrrrrr|r}
%             & x_1 & x_2 & x_3 & x_4 & x_5 & x_6 &     \\ \hline
%         -Z  & -4  & 0   & 1   & 0   & -2  & 0   & -50 \\ \hline\hline
%         x_4 & 1   & 0   & 0   & 1   & 0   & 0   & 5   \\
%         x_2 & 4   & 1   & 0   & 0   & 1   & 0   & 5   \\
%         x_6 & -8  & 0   & 1   & 0   & -4  & 1   & 25  \\ \hline
%     \end{array}
% $$
% Αλλάζουμε την μεταβλητή $x_6$ με την $x_3$.
% $$
%     \begin{array}{r|rrrrrr|r}
%             & x_1 & x_2 & x_3 & x_4 & x_5 & x_6 &     \\ \hline
%         -Z  & 4   & 0   & 0   & 0   & 2   & -1  & -75 \\ \hline\hline
%         x_4 & 1   & 0   & 0   & 1   & 0   & 0   & 5   \\
%         x_2 & 4   & 1   & 0   & 0   & 1   & 0   & 25  \\
%         x_3 & -8  & 0   & 1   & 0   & -4  & 1   & 25  \\ \hline
%     \end{array}
% $$
% Αλλάζουμε την μεταβλητή $x_4$ με την $x_1$.
% $$
%     \begin{array}{r|rrrrrr|r}
%             & x_1 & x_2 & x_3 & x_4 & x_5 & x_6 &     \\ \hline
%         -Z  & 0   & 0   & 0   & -4  & 2   & -1  & -95 \\ \hline\hline
%         x_1 & 1   & 0   & 0   & 1   & 0   & 0   & 5   \\
%         x_2 & 0   & 1   & 0   & -4  & 1   & 0   & 5   \\
%         x_3 & 0   & 0   & 1   & 8   & -4  & 1   & 65  \\ \hline
%     \end{array}
% $$
% Αλλάζουμε την μεταβλητή $x_2$ με την $x_5$.
% $$
%     \begin{array}{r|rrrrrr|r}
%             & x_1 & x_2 & x_3 & x_4 & x_5 & x_6 &      \\ \hline
%         -Z  & 0   & -2  & 0   & 4   & 0   & -1  & -105 \\ \hline\hline
%         x_1 & 1   & 0   & 0   & 1   & 0   & 0   & 5    \\
%         x_5 & 0   & 1   & 0   & -4  & 1   & 0   & 55   \\
%         x_3 & 0   & 4   & 1   & -8  & 0   & 1   & 85   \\ \hline
%     \end{array}
% $$
% Αλλάζουμε την μεταβλητή $x_1$ με την $x_4$.
% $$
%     \begin{array}{r|rrrrrr|r}
%             & x_1 & x_2 & x_3 & x_4 & x_5 & x_6 &      \\ \hline
%         -Z  & -4  & -2  & 0   & 0   & 0   & -1  & -125 \\ \hline\hline
%         x_4 & 1   & 0   & 0   & 1   & 0   & 0   & 5    \\
%         x_5 & 4   & 1   & 0   & 0   & 1   & 0   & 25   \\
%         x_3 & 8   & 4   & 1   & 0   & 0   & 1   & 125  \\ \hline
%     \end{array}
% $$
% \subsection*{(β)}
% \begin{figure}[h]
%     \includegraphics[width=\textwidth]{q6_b.jpg}
%     \caption{Γράφος με τα βήματα του αλγορίθμου}
%     \label{fig:q6_b}
% \end{figure}
% \clearpage
% Κώδικας για την άσκηση 6.
% \selectlanguage{english}
% \lstinputlisting[language=Python]{src/question6.py}
% \selectlanguage{greek}
\end{document}